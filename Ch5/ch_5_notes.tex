\documentclass{tufte-handout}
\usepackage[utf8]{inputenc}
\usepackage[T1]{fontenc}
\usepackage{listings}

\title{Head First Java: Chapter 5 Notes}
\lstset{language=java, basicstyle={\small\ttfamily}, breaklines=true}

\begin{document}

    \maketitle

    \section*{Flow and Principles of Java Programming}
    Here a simple Java program will be used to example the flow of writing Java programs. This program is a simplified version of the ``Battleship'' game called ``Simple Dot Com'', but in two dimension (a row), and there is only one 3-cell long target. It is recommended when writing Java code, the coder follow these steps:
    \begin{enumerate}
        \item Prep Code --- This is essentially the pseudo-code for Java. Most prep code contains three parts:
        \begin{enumerate}
            \item Instance Variable Declarations. For example:
            
            \texttt{DECLARE an int array to hold the location cells. Call it locationCells.\newline
            DECLARE an int to hold the number of hits. Call it numOfHits and SET it to 0.}
            
            \item Method Declarations. For example:

            \texttt{DECLARE a checkYourself() method that takes a String for the user's guess, checks it and return a result representing a ``hit'', ``miss'', or ``kill''.\newline
            DECLARE a setLocationCells() setter method that takes an int array which has the three cell locations as ints.}

            \item Method Logic. For example for the above two methods, the method logic prep code would be:
            \begin{lstlisting}
METHOD String checkYourself(String userGuess)
    GET the user guess as a String parameter
    CONVERT the user guess to an int
    REPEAT with each of the location cells in the int array
        COMPARE the user guess to the location cell
        IF the user guess matches
            INCREMENT the number of hits
            FIND OUT if it was the last location cell
            IF number of his is 3
                RETURN ``Kill'' as the result
            ELSE it was not a so so
                RETURN ``Hit''
            END IF
        ELSE the user guess did not match so
            RETURN ``Miss''
        END IF
    END REPEAT
END METHOD

METHOD void setLocationCells(int[] cellLocations)
    GET the cell locations as in int array parameter
    ASSIGN the cell locations parameter to the cell locations in stance variable
END METHOD
            \end{lstlisting}
        \end{enumerate}
        
        Based on this prep code, we can determine the \emph{test code}.
        \item Test Code --- Writing the test code before the real code helps with proper code structure. What the test code should test is self-evident from the prep code. For example, for the \texttt{checkYourself()} method above:
        \begin{enumerate}
            \item Instantiate a \texttt{SimpleDotCom} object (this is the target and will hold the \texttt{main} method).
            \item Assign the \texttt{SimpleDotCom} object a location (the \texttt{locationCells} instance variable), which is an array of 3 \texttt{int}.
            \item Create a String (\texttt{userGuess}) to represent a user guess.
            \item Invoke \texttt{checkYourself()} with fake user guess as argument.
            \item Print out the result to see if it is correct.
        \end{enumerate}

        Note that because the actual program (\textit{i.e.} the \emph{real code}) hasn't been written yet, the test code written beforehand probably would not compile or run. But writing the test code first helps with structuring the actual program later. Given the outline above, the test code would be:
        \begin{lstlisting}
public class SimpleDotComTestDrive {
    public static void main (String[] args) {
        SimpleDotCom dot = new SimpleDotCom();
        int[] locations = {2, 3, 4};
        dot.setLocationCells(locations);
        String userGuess = ``2'';
        String result = dot.checkYourself(userGuess);
        String testResult = ``failed'';
        if (result.equals(``hit'')) {
            testResult = ``passed'';
        }
        System.out.println(testResult);
    }
}
        \end{lstlisting}

        \item Real Code --- Once the test code is written, that should give an idea of what is needed in the real code. Based on the scaffolding of the prep code, the actual code of the program can be written. For example, the test code included the \texttt{checkYourself();} method that check to see if the user guess hit the cell location:
        \begin{lstlisting}
public String checkYourself(String stringGuess) {
    int guess = Interger.parseInt(stringGuess);
    String result = ``miss'';
    for (int cell : locationCells) {
        if (guess == cell) {
            result = ``hit'';
            numOfHits++;
            break;
        }
    }
    if (numOfHits == locationCells.length) {
        result = ``kill'';
    }
    System.out.println(result);
    return result;
}
        \end{lstlisting}
    \end{enumerate}

    \section*{For Loops}
    There are two types of \texttt{for} loops in java:

    \begin{enumerate}
        \item Regular For Loops --- These are primitive for loops that manually tells Java how many times to loop:
        \begin{lstlisting}
for(int i = 0; i < 100; i++){}             
        \end{lstlisting}
        For example the above for loop tells Java to repeat code within the \texttt{\{\}} for 100 times. There are three parts of these regular for loops
        \begin{enumerate}
            \item Initialization --- Declare and initialize a variable to use within the loop body (\textit{i.e. a counter}). More than one variable can be initialized here. \texttt{int i = 0;} 
            \item Boolean Test --- The condition in which the loop will end. Because Java evaluates this as a Boolean test i must resolve into \texttt{true} or \texttt{false}. \texttt{i < 100;}
            \item Iteration Expression --- At the end of each loop, whatever written here will happen. \texttt{i++;}
        \end{enumerate}
        \item Enhanced For Loops --- Only available Java 5.0+. These are conceptually similar to Python for loops that iterates over an array (Python lists, etc.):
        \begin{lstlisting}
for (String name : nameArray) {}                     
        \end{lstlisting}
        This for loop loops over each element in the array \texttt{nameArray}. There are two parts to an enhanced for loop:
        \begin{enumerate}
            \item Iteration Variable Declaration --- Declare and initiate a variable to use within the loop body. Similar in concept with Python loops, except the type must also be specified (usually String, but really can be any types.) 
            \item Iterable Collection --- The array for which to iterate over. Non-array collections can also be used here. 
        \end{enumerate}         
    \end{enumerate}

    \section*{Casting Primitives and Implicit Widening}
    Recall that different variable types have different bit depth. Implicit Widening refer to when a smaller bit depth variable is assigned to a larger bit depth variable or placed into an array of a larger bit depth variable. For example:
    \begin{lstlisting}
byte x = 4;
int y = 7;
y = x;

jshell> y ==> 4
    \end{lstlisting}
    Here, a x is a \texttt{byte} type variable, which is a shallower bit depth than y which is an \texttt{int} type variable. Therefore when the value of x was assigned to y, this is a legal operation because a byte ``fits'' into an int, due to implicit widening. The opposite is not true:
    \begin{lstlisting}
byte x = 4;
int y = 7;
x = y;

jshell> Error:
|  incompatible types: possible lossy conversion from int to byte
|  x = y
|      ^     
    \end{lstlisting}
    Java raised an exception here because an int cannot ``fit'' into a ``byte''.

    In order for a variable with a deeper bit depth to ``fit'' into a shallower one, the variable has to be explicitly \emph{cast} into a variable type. Consider the same example:
    \begin{lstlisting}
byte x = 4;
int y = 7;
x = (byte) y;
jshell> x ==> 7    
    \end{lstlisting}
    The line of code \texttt{x = (byte) y;} tells the Java compiler to \emph{cast} the value of y (in int type) into a byte type (the type for x) by truncating the value to fit into the smaller type. This can lead to unexpected behavior if the variable value that is being cast is larger than the maximum size of the variable type being cast to. Finally, variable casting is a convenient way to truncate decimal numbers to whole numbers by casting either a double or a float to a byte/int/short/long.

\end{document}