\documentclass{tufte-handout}
\usepackage[utf8]{inputenc}
\usepackage[T1]{fontenc}
\usepackage{listings}

\title{Head First Java: Chapter 6 Notes}
\lstset{language=java, basicstyle={\small\ttfamily}, breaklines=true}

\begin{document}

    \maketitle

    \section*{The ArrayList Pre-Built Class}
    The \texttt{ArrayList} class in Java is a built-in class that offers an enhanced array-object that has functionality similar to lists and other collections in Python. The \texttt{ArrayList} class differs from regular Java arrays by:

    \begin{itemize}
        \item The size of the \texttt{ArrayList} object is mutable, it does not need to be specified during declaration. Whereas for a regular array the size must be specified when declared. E.g. \texttt{new String[2];}
        \item When adding objects into an \texttt{ArrayList}, specifying an index is optional: \texttt{myArrayList.add(object);} is valid as is \texttt{myArrayList.add(0, object);} In an regular array the index must be specified.
        \item It is possible to remove elements in an \texttt{ArrayList} via the \texttt{myArrayList.remove();} method. The size of the \texttt{ArrayList} will adjust to the updated size. This is not possible with regular arrays; the most one can do is to set an element to \texttt{null}.
        \item While the type on a regular array needs to be specified during declaration, this is done differently with \texttt{ArrayList}. The element type of the \texttt{ArrayList} should be \emph{parameterized} during declaration. For example, to declare an \texttt{ArrayList} with \texttt{String} elements: \texttt{ArrayList<String> myArrayList = new ArrayList<String>;}
        \item The square bracket syntax for regular arrays (e.g. \texttt{myArray[0]}) is not used in \texttt{Arraylist}. To access specific element by index in an \texttt{ArrayList} use the \texttt{myArrayList.get();} method.
    \end{itemize}

    To use \texttt{ArrayList} in Java, it must be imported via \texttt{import java.util.ArrayList}.
    
\end{document}